\documentclass[12pt]{article}
\usepackage[utf8]{inputenc}
\usepackage{eurosym}
\usepackage{textcomp}
\usepackage[document]{ragged2e}
\usepackage{geometry, array, graphicx, wrapfig, float, amsmath, titlesec, amsmath, amsfonts, mathtools, amssymb, colortbl, babel}
\usepackage[thinlines]{easytable}
\graphicspath{ {./pics/} }
\geometry{a4paper, inner=3.5cm, outer=2.5cm, bottom=2.5cm, top=2.5cm}
\renewcommand{\baselinestretch}{1.5}
\renewcommand{\arraystretch}{0.8}
\setlength{\parskip}{1em}
\makeatletter
\makeatother
\begin{document}
	\newcommand{\sectionbreak}{\clearpage}
	\begin{titlepage}
		\begin{minipage}{0.36\textwidth}
			\begin{figure}[H]
				\includegraphics[scale=0.1]{elte_cimer_szines.jpg}
			\end{figure}
		\end{minipage}
		\begin{minipage}{0.60\textwidth}
			\begin{center}
				{\Large Eötvös Loránd Tudományegyetem \\
					Informatikai Kar \\
					Komputeralgebra Tanszék \\}
			\end{center}
		\end{minipage}
		\\[0.3\baselineskip]
		\noindent\makebox[\linewidth]{\rule{\textwidth}{0.5pt}}
		\vspace*{\fill}
		\centering
		\vspace*{0.5cm}
		
		\huge\bfseries
		Pszeudovéletlen sorozatok mértékei és konstrukciói		
		\vspace*{0.5cm}
		
		\vspace*{\fill}
		\begin{TAB}(r,0.5cm,0.5cm)[9pt]{cc}{cccc}
			\normalfont \normalsize \textsl{ Szerző: } & \normalfont \normalsize \textsl{ Témavezető: } \\
			\normalsize Kovács Levente & \normalsize Dr. Tóth Viktória \\
			\normalfont \normalsize programtervező informatikus BSc & \normalfont \normalsize egyetemi adjunktus \\
			\normalsize & \normalfont \normalsize Komputeralgebra Tanszék
		\end{TAB}
		\\
		\large \normalfont Budapest, 2019
	\end{titlepage}
	\renewcommand*\contentsname{Tartalomjegyzék}
	\tableofcontents
	\newpage
	\section{Bevezető}
	\subsection{Történeti betekintés}
	Véletlen sorozatok generálására sok modern területnek igénye van (pl.: statisztika, szimulációk, kriptográfia). Számítógéppel azonban nem lehetséges valódi véletlen számokat generálni, ezért pszeudovéletlen eljárásokra hagyatkozunk. Cél az, hogy olyan eljárásokat használjunk melyek statisztikailag véletlennek tűnnek, attól függetlenül, hogy egy determinisztikus rendszer generálja az eredményt.
	\par
	Az ilyen sorozatokra való igény először Gilbert Vernam munkájának következménye. 1917-ben Vernam feltalálta a one-time pad titkosítási eljárást. Az üzeneteket telegráfon bitenként továbbították, viszont az eredeti adatot valamilyen kulcs segítségével megváltoztatták (amit aztán a fogadó fél a kulcs ismeretében vissza tud fejteni). Ha le is hallgatták a küldött adatot, akkor a kulcs nélkül nem lehetett tudni, hogy mi volt az eredeti üzenet.
	\par
	1949-ben Claude Shannon, matematikus, bebizonyította a fenti eljárásról, hogyha egy jó sorozatot használunk kulcsnak, akkor az eredeti és a titkosított bitsorozat között nem lehet semmiféle összefüggést találni. Ennek következménye, hogy a Vernam-féle one-time pad eljárást nem lehet feltörni, ha megfelelő, egyszer használatos kulcsokat használunk.
	\par
	Az is igaz azonban, hogyha nem megfelelő kulcsokat használunk (pl. többször használjuk a kulcsokat), akkor a titkosítás könnyen törhető. A fő nehézség tehát a one-time pad használatánál, hogy nagy méretű véletlen bitsorozatokat kell előállítani kulcsoknak.
	\par
	Az ilyen sorozatok előállításához régen fizikai eszközöket használtak, pl. diódákat. A problémája az ilyen eszközöknek, hogy külső tényezők is befolyásolják a véletlenséget, és meghibásodás is történhet. Ezért nagyon fontos, a véletlenség vizsgálása különböző statisztikai tesztekkel, ami azonban időigényes. Emiatt jobb, hogyha "csak" pszeudovéletlen sorozatokat állítunk elő. Így elkerülhető a statisztikai tesztelés, mivel bizonyítottan jó véletlen tulajdonságokkal rendelkező sorozatot kapunk.
	\par
	A pszeudovéletlen tulajdonságok mérésével sok évtizede foglalkoznak matematikusok. Elsőként Émile Borel definiálta végtelen bináris sorozatokra a pszeudovéletlenség normalitás mértékét. Később Solomon Wolf Golomb és Donald Knuth próbálkoztak a pszeudovéletlenség matematikai fogalmának definiálására. Ennek a definíciónak alapvető gyengeségei voltak, ezért ezt ma már nem nagyon használják. Ezután Andrej Nyikolajevics Kolmogorov és Gregory John Chaitin bonyolultságelméleti irányból közelítették meg a problémát, de a gyakorlatban ezek a fogalmak nem kaptak nagy jelentőséget, mivel alkalmazásuk nem jól kivitelezhető.
	\par
	Ezen próbálkozások inspiráltak egy új, konstruktív megközelítést a pszeudovéletlenség vizsgálatára. Christian Manduit és Sárközy András megközelítése az volt, hogy léteznek olyan sorozatok, melyekről bizonyítható, hogy statisztikailag erős véletlen tulajdonságokkal rendelkeznek (más szóval: erős pszeudovéletlen tulajdonságúak), így az utólagos tesztelés elkerülhető. Ez a fajta vizsgálat az évek során egyre szélesebb körben alkalmazott.
	\subsection{A program célja}
	A programom célja, hogy a fent említett konstrukció keretén belül vizsgáljak különböző pszeudovéletlen sorozatokat. A vizsgált konstrukciókat két csoportból válogattam. Egyik kategóriába matematikai hátterű sorozatok tartoznak, melyeknek működése főként számelméleti ismereteken alapulnak. A másik csoport az iparban széles körben használt pszeudovéletlen generátorok, amik nem feltétlenül matematikai hátterűek.
	\par
	A programban lehetőség nyílik a pszeudovéletlen konstrukciók kipróbálására, és a generált sorozatok vizsgálatára a statisztikai mértékek szerint. A sorozatokat ki is próbálhatjuk a Vernam-féle one-time pad titkosítási eljárással.
	\newpage
	\section{Felhasználói dokumentáció}
	A program pszeudovéletlen sorozatkonstrukciók kipróbálására és a sorozatok vizsgálatára készült a statisztikai mértékekkel.
	\subsection{Rendszerkövetelmények}
	\subsection{Futattás}
	\subsection{Menürendszer}
	A menürendszer pontjainak 3 fő csoportja van, ezek:
	\begin{itemize}
		\item Alapvető funkciók: Kilépés, vissza a főmenübe
		\item Konstrukciók: A konstrukciók kipróbálása
		\item Tesztelés: Mértékek és titkosítás
	\end{itemize}
	\subsubsection{Főmenü}
	Amikor a program elindul a felhasználót a főmenü fogadja. Itt elérhető az összes megvalósított konstrukció almenüpontja és a mértékek, illetve titkosításé is.
	
	Kilépni a program jobb felső sarkában található "X" gombbal, vagy a főmenüben található "Kilépés" gombbal lehet.
	\subsection*{Konstrukciók}
	Ez a pont csak alapvető segítséget nyújt a különböző konstrukciók kipróbálásához. Részletes leírása az algoritmusoknak a fejlesztői dokumentációban találhatóak.
	\par
	\subsubsection{Legendre szimbólumos konstrukció}
		Ez a konstrukció a Legendere szimbólumon alapul, melynek véletlen tulajdonságait régóta vizsgálják. Mauduit és Sárközy ezen a sorozaton vizsgálta először 1997-ben a véges bináris sorozatokra készített jól-eloszlás és korreláció-mértékeket [C. Mauduit, A. Sárközy, On finite pseudorandom binary sequences I: The measures of pseudorandomness, the Legendre symbol].
	\par
	Szükséges paraméterek:
	\begin{enumerate}\bfseries
		\item A sorozat hossza
		\\ \normalfont{A sorozat hosszát bájtokban kell megadni, tehát az 1 bájt hosszú sorozat 8 bitet foglal magába. Bármennyi lehet.}
		\bfseries \item Prímszám
		\\ \normalfont Egy olyan prímszámra van szükségünk a konstrukcióhoz, mely legalább 16-szor nagyobb mint a sorozat hossza (bitben megadott méret esetén 2-szer nagyobb). Az is kell, hogy a 2 legyen primitív gyök modulo prím. A "Generálás" gombbal a program megkeresi a legkisebb olyan prímet mely a fenti feltételeknek megfelel. A "Következő" gombbal ugorhatunk a következő legkisebb jó prímre.
		\bfseries \item A konstrukcióhoz használt polinom fokszáma \\ \normalfont
		A fokszám nem lehet túl nagy függően a prímszámtól, de túl kicsi sem. Az alsó határ jelen megvalósításban 2-nek, a felső $5\cdot\sqrt[\leftroot{-2}\uproot{2}10]{p}$-nek lett megszabva, ahol p a használt prímszám. Polinom fokszámot generálhatunk a fenti intervallumban a "Generálás" gombbal.
		\bfseries \item A polinom \\
		\normalfont
		A konstrukció jó működéséhez szükséges, hogy olyan polinomot válasszunk, amelynek csak egyszeres gyökei vannak. Emiatt a polinomot a gyökeinek felsorolásával reprezentáljuk, ezeket a szóközzel elválasztva kell felsorolni a szövegmezőben. A fokszámmal megegyező számú gyököt kell megadni. Lehetőség van polinom generálására is, ekkor a program legenerál egy az előző pontban megadott fokszámú polinomot.
	\end{enumerate}
	Ha fenti paramétereket megadtuk, akkor a "Sorozat generálása" gombra kattintva megkapjuk az eredményét a Legendre konstrukciónak a kapott bemenetre.
	\subsubsection{RC4 konstrukció}
	Az RC4 konstrukciót Ron Rivest fejlesztette ki az 1987-ben viszont eredetileg üzleti titok volt. 1994-ben azonban kiszivárgott az algoritmus leírása. A működése egyszerű és gyors, ezért használata széles körben elterjedt, viszont vannak rossz tulajdonságai.
	\par
	Szükséges paraméterek:
	\begin{enumerate}
		\bfseries\item A sorozat hossza \\
		\normalfont A sorozat hosszát bájtokban kell megadni. Bármennyi lehet.
		\bfseries \item A kulcs
		\\
		\normalfont A kulcs változó méretű lehet, ez alapján fog történni a generálás. Tipikusan 40-2048 bit hosszú a kulcs. Ez a megvalósítás egy szöveget vár kulcsnak, így átfogalmazva a feltételt legalább 5, maximum 256 karakter hosszú lehet a kulcs.
	\end{enumerate}
	A fenti paraméterek megadása után megtekinthetjük az RC4 konstrukció által generált sorozatot a "Sorozat generálása" gombbal.
	\subsubsection{Additív karakteres konstrukció}
	Ezt a konstrukciót Mauduit, Rivat és Sárközy vezették be  [C.Mauduit, J.Rivat, A.Sárközy, Construction of pseudorandom binary sequences using additive characters]. \par
	Szükséges paraméterek:
	\begin{enumerate}
		\bfseries \item A sorozat hossza \\
		\normalfont A sorozat hosszát bájtokban kell megadni. Bármennyi lehet. \\
		\bfseries \item Prímszám \\
		\normalfont Ebben a konstrukcióban a prímszám nagyságára nincs igazi nagyságrendi követelmény. A "Generálás" gombbal a legkisebb, sorozat hosszánál nagyobb, prímszámot adja vissza. A "Következő" gomb megkeresi az első jelenleginél nagyobb prímet.
		\bfseries \item A konstrukcióhoz használt polinom fokszáma \\
		\normalfont Hasonlóan a Legendre konstrukcióhoz, a fokszámtól elvárjuk, hogy ne legyen túl nagy vagy túl kicsi. Az alsó határ jelen megvalósításban 2-nek, a felső $5\cdot\sqrt[\leftroot{-2}\uproot{2}10]{p}$-nek lett megszabva, ahol p a használt prímszám. A "Generálás" gombbal egy fokszámot kapunk a fenti intervallumból.
		\bfseries \item A polinom \\
		\normalfont A Legendre konstrukcióhoz hasonlóan fontos, hogy a polinomnak csak egyszeres gyökei legyenek. Ezért a polinomot legenerálhatjuk úgy, hogy a gyökeit soroljuk fel, szóközzel elválasztva. A fokszámmal megegyező számú gyököt kell megadni. Lehetőség van polinomot generálni a fenti követelményeknek eleget téve a "Polinom generálása" gombbal.
	\end{enumerate}
	\subsubsection{ChaCha20 konstrukció}
	A ChaCha20 egy stream cipher eljárás, mely a Salsa20 egy változata. Mindkét algoritmust Daniel J. Bernstein német-amerikai matematikus, kriptológus fejlesztette ki. A Salsa változatot 2005-ben tervezte, 2007-ben publikálták először [https://cr.yp.to/snuffle/salsafamily-20071225.pdf], az eSTREAM projekt keretében belül. A ChaCha változat 1 évvel később, 2008-ban lett publikálva, célja az volt, hogy növelje a lavinahatást és ugyanolyan, vagy picit jobb teljesítményű legyen mint a Salsa.
	
	Mindkét eljárás eszközei a bitenkénti XOR (kizáró vagy), a 32-bites összeadás $\textrm{mod}\ 2^{32}$, és a konstans távolságú forgatás egy belső állapoton, ahol az állapot 16 db 32-bites adatból (szóból) áll. Csak ezen műveletekkel az algoritmus kivédi az időzítéses támadásokat. Ez olyan támadás, ahol a támadó fél a futtató rendszer hiányosságait kihasználva (tehát nem a szoftver implementációja a baj) a kriptográfiai algoritmus futási idejéből próbálja feltörni a rendszert, mivel a bemenettől függhet a futási idő. Csak fix 32-bites műveleteket használva a futási idő megegyezik minden bemenetre, mivel az algoritmus fixen 20 menetben végzi fel a fenti műveleteket és a műveletek száma is fix minden menetben.
	
	A ChaCha algoritmus állapotrendszerének és algoritmusa részletes leírása a fejlesztői dokumentációban található. Jelen megvalósítás az RFC 8439-ben [https://tools.ietf.org/html/rfc8439] definiáltakat követi melyet az IRTF publikált 2018-ban, de ezen kívül több változata van az algoritmusnak, főként a körök számára vonatkozóan.
	
	A szükséges bemenetek:
	\begin{enumerate}
		\bfseries \item A sorozat hossza \\
		\normalfont A sorozat hosszát bájtokban kell megadni. Bármennyi lehet.
		\bfseries \item Kezdőállapot \\
		\normalfont A kezdőállapot 16 darab 32 bites szóból áll, jelen implementációban az egyszerűbb használat miatt 16 darab előjel nélküli 32 bites egész számot jelent. Az állapot 4 részre bomlik, ebből az egyik kategória konstans, 4 darab szóval, így ezt nem kell megadni. A maradék 3 csoport 12 szavát kategóriánként kell megadni:
		\begin{enumerate}
			{\itshape \item {Kulcs}} \\
			Ez 8 darab állapotot fed le. Jelen esetben 8 darab számot kell megadni a $[0, 2^{32}-1]$ intervallumban. Tudunk kulcsot generálni a fenti intervallumból vett számokkal a "Generálás" gombra kattintva.
			{\itshape \item {Számláló}} \\
			A jelen megvalósításban ez egy szót fed le (van verzió, ahol 64 bites a számláló, vagyis 2 szót foglal le), a megvalósításban ez minden "körében" az algoritmusnak növelődik, mondhatni számolja az iterációkat. Egy számot kell megadni a $[0, 2^{32}-1]$ intervallumban.
			{\itshape \item {Egyszer használatos kulcs (nonce)}} \\ 
			Ez a maradék 3 szót foglalja magában. Ez a kulcs azért van külön kategóriában mert ezt a kulcsot csak egyszer lehet használni egy titkosított kommunikációban, hogy kivédje az újraküldéses támadásokat. Ez általában egy tetszőleges szám, de mindig más. Jelen megvalósításban ez 3 darab szám megadása a $[0,2^{32}-1]$ intervallumban. Generálhatunk ilyen egyszer használatos kulcsot a fenti intervallumból vett számokkal a "Generálás" gombra kattintva.
		\end{enumerate}
	\end{enumerate}
	\subsection{Mértékek}
	Ezen része a programnak az átlagos felhasználónak nem feltétlenül tud sokat nyújtani. Ahhoz, hogy a mértékek eredményeit értelmezni tudjuk szükséges a formulákról és a jelentésükről való ismertség megszerzése. Ezen ismeretek alapjait a fejlesztői dokumentációban részletezem, a dokumentáció ezen részén csak röviden foglalom össze a program funkcióit.
	
	Három fő mértékünk van:
	\begin{enumerate}
		\bfseries \item Jól-eloszlás
		\bfseries \item Normalitás 
		\bfseries \item Korreláció
	\end{enumerate}
	Ezekből kettőt közvetlen ki tudunk számolni, de a harmadikra csak közelítő értéket ad a program, mivel a számolás műveletigénye olyan nagyságrendeket ér el, hogy ennek kivárása rengeteg időbe telne a szakdolgozat keretein belül.
	
	A számolási paraméterek:
	\begin{enumerate}
		\bfseries \item A vizsgálandó sorozat \\
		\normalfont A sorozat, melyre ki szeretnénk számolni a mértékeket. Korábban mentett sorozat betölthető a "Sorozat betöltése" gombbal, de saját magunk is írhatunk be tetszőleges sorozatot. \\
		A jól-eloszlás és a normalitás mértékek számolásához további paraméter nem szükséges, ezeket kiszámolhatjuk a "Számolás" gombbal. \\
		\bfseries \item A k-ad rendű korreláció közelítésének rendje és menetszáma \\
		\normalfont Ennek a mértéknek a számolása, mint fent említettem, túlságosan nagy műveletigényű a dolgozat keretein belül, ezért csak közelítő eredményt adunk a k-ad rendű korrelációra (gyakorlatilag ezeknek a maximuma a korreláció mérték). Meg kell adnunk, hogy hányad rendű korrelációt szeretnénk vizsgálni, és hogy hány menetes a számolás (hányszor próbálkozik új sorozattal). Hasonlóan a többi mértékhez, a "Számolás" gombbal tudjuk elkezdeni a műveletet.
	\end{enumerate}
	\subsection{One-time pad titkosítás}
	A One-time pad eljárásnak széleskörű alkalmazásai vannak a titkosításokban. Gilbert S. Vernam fejlesztette ki, és elméletben tökéletes titkosságot nyújt megfelelően véletlen bemenő paraméter esetén.
	
	Használatához először egy titkosítandó szöveget kell megadni, mely a magyar billentyűzet legtöbb betűjét tartalmazza. Kivételek a ritkán használt szimbólumkarakterek mint pl. a '˙', teljes listát a fejlesztői dokumentáció tartalmazza. A titkosításhoz használt bitsorozatot kell még megadni, mely fontos, hogy legalább olyan hosszú legyen, mint a bemenő szöveg karaktereinek számának 7-szerese. Eredményként a titkosított szöveget kapjuk a fenti karakteres formában.
	
	Bemenő paraméterek:
	\begin{enumerate}
		\bfseries \item Titkosítandó szöveg \\
		\normalfont A szöveget amit titkosítani szeretnénk, a magyar billentyűzet legtöbb karakterét tartalmazhatja. Hosszúsága bármennyi lehet.
		\bfseries \item Titkosító bitsorozat \\
		\normalfont A bitsorozat mely a titkosítás kulcsaként szolgál. Betölthető korábban lementett sorozat a "Sorozat betöltése" gombbal, de magunk is írhatunk be sorozatot. Fontos, hogy a bitsorozat hossza legalább annyi legyen, mint a titkosítandó szöveg karakterszámának 7-szerese. A titkosítást különben nem lehet elvégezni. \\
		Ezután a "Szöveg titkosítása" gombra kattintva lefuttathatjuk a titkosítást és megtekinthetjük az eredményt a "A titkosított szöveg" fülnél. Ezt a szöveget visszafejthetjük az eredetivé, ha az eredeti kulcsként használt bitsorozattal újra lefuttatjuk a titkosítást a titkosított szövegen.
	\end{enumerate}
	\newpage
	\section{Fejlesztői dokumentáció}
	\subsection{Konstrukciók}
	\subsubsection{Legendre konstrukció}
	\subsubsection*{Ismertető}
	\textbf{Def.}: Ha $p$ egy prímszám, és $a \in \mathbb{Z}$, akkor az $\left({\frac{a}{p}}\right)$ Legendre-szimbólum értéke: \par
	$$\left({\frac{a}{p}}\right) =
	\begin{cases*}
	1 & ha $a$ kvadratikus maradék modulo $p$ és $a \not\equiv 0\ (\textrm{mod}\ p)$ \\
	-1       & ha $a$ nem kvadratikus maradék modulo $p$\\
	0 & ha $a \equiv 0\ (\textrm{mod}\ p)$
	\end{cases*}$$
	A Legendre-szimbólum véletlenségét már régóta vizsgálták, ez volt a motiváció a konstrukció megvalósítására.
	\subsubsection*{Algoritmus}
	\begin{enumerate}
		\item Vegyünk egy olyan $p$ prímet, hogy a $2$ legyen primitív gyök modulo $p$
		\item Vegyünk egy olyan $f \in \mathbb{F}_p[x]$ polinomot, melynek csak egyszeres gyökei vannak, és foka nem túl nagy p-hez képest
		\item $k$ hosszú sorozat esetén a ($e_0, e_1, ..., e_{k-1}$) sorozat $n$. tagja: $$e_n = 
		\begin{cases*}
		 \left({\frac{f(i)}{p}}\right) &, ha $f(i) \nmid  p$  \\
		1 &, ha $f(i) \mid p$ 
		\end{cases*}
		$$
	\end{enumerate}
	\subsubsection*{Extra követelmények}
	Ha egy $n$ hosszú sorozatot szeretnénk generálni, akkor a használt $p$ prímre legyen igaz, hogy $p > 2n$. A polinom fokszáma szeretnénk, ha nem lenne túl kicsi, és a felső határt $p$-hez viszonyítva kell megadni. Jelen megvalósításban ha $k$ a polinom fokszáma, akkor legyen $2 \leq k \leq 5\cdot\sqrt[\leftroot{-2}\uproot{2}10]{p}$.
	\subsubsection{RC4 konstrukció}
	\subsubsection*{Ismertető}
	Az RC4 stream cipher algoritmus a bájt hosszúságú bitsorozatok egy permutációjának előállításával kezdődik, ami egy kulcssorozat segítségével történik. Ez a key-scheduling fázisa az algoritmusnak.
	\par
	Ez után következik a tényleges sorozat generálás. Ameddig kell a sorozatot generálni (pl. hossz alapján), addig az algoritmus lépésenként keveri a bájtok sorrendjét, és közben egy bájtnyi kimenetet generál.
	\par
	Ebben a megvalósításban a klasszikus, módosítások nélküli algoritmust valósítottam meg, hogy ez hogyan viszonyul a megvalósított mértékekhez.
	A módszernek több változata van, amik javíthatják a véletlenségét a generált sorozatnak.
	\subsubsection*{Algoritmus}
	\begin{enumerate}
		\item Töltsünk fel egy 256 elemű $S$ tömböt úgy, hogy $i$. indexére: $S[i]:=i$. 0-tól indexelünk, és gyakorlatilag az összes 8 hosszúságú bitsorozatot kell előállítani
		\item Adjunk meg egy $n$ méretű $K$ (ebben a megvalósításban egy $n$ karakterű ASCII kódolású szöveg) kulcstömböt, ahol $1 \leq n \leq 256$
		\item Ez az algoritmus key-scheduling fázisa. Legyen $i, j:=0$, majd:
			\begin{enumerate}
				\item Legyen $j:=(j+S[i]+K[i \ \textrm{mod}\ n]) \ \textrm{mod} \ 256$
				\item Cseréljük ki $S[i]$ és $S[j]$ értékét
				\item Inkrementáljuk $i$-t, ha $i < 256$, akkor ugorjunk vissza (a)-ra
			\end{enumerate}
		\item Ez a sorozatgeneráló fázis. Legyen $i, j := 0$, $n$-szer kell generálni 1 bájtnyi kimenetet, tehát ennek követésére legyen $l:=0$, majd:
		\begin{enumerate}
			\item Legyen: $i := (i+1) \ \textrm{mod} \ 256$, majd $j := (j + S[i]) \ \textrm{mod} \ 256$
			\item Cseréljük ki $S[i]$ és $S[j]$ értékét
			\item Legyen $T := S[(S[i] + S[j]) \ \textrm{mod} \ 256]$, T lesz a mostani generálási lépés 1 bájtnyi kimenete
			\item Inkrementáljuk $l$-t, ha $l < n$, akkor ugorjunk vissza (a)-ra
		\end{enumerate}
	\end{enumerate}
	\subsubsection*{Extra követelmények}
	Elvárjuk, hogy a kulcssorozat ne legyen túl rövid (általában 40 bit az alsó határ). Jelen megvalósításban: $n \geq 5$, vagyis legalább 5 karakter hosszú kulcs kell.
	\subsubsection{Additív karakteres konstrukció}
	\subsubsection*{Bevezető}
	Az additív karakteres konstrukció a Legendre szimbólumoshoz áll legközelebb. A Legendre szimbólumos konstrukció jobb véletlenségi tulajdonságokat mutat. Az additív karakteres sorozat azonban nagyobb sorozatcsaládot képes előállítani, és implementációja egyszerűbb.
	
	Nevét a konstrukció jó tulajdonságait bizonyító tételben használt additív karakterekről kapta.
	\subsubsection*{Algoritmus}
	\begin{enumerate}
		\item Vegyünk egy páratlan $p$ prímet
		\item Vegyünk egy olyan $f \in \mathbb{F}_p[x]$ polinomot, melynek csak egyszeres gyökei vannak, és foka nem túl nagy $p$-hez képest
		\item $k$ hosszú sorozat esetén a ($e_0, e_1, ..., e_{k-1}$) sorozat $n$. tagja: $$e_n = 
		\begin{cases*}
		1 &, ha $f(i) \ \textrm{mod} \ p  < \frac{p}{2}$  \\
		0 &, ha $\frac{p}{2} \leq f(i) \ \textrm{mod} \ p$ 
		\end{cases*}
		$$
	\end{enumerate}
	\subsubsection*{Extra követelmények}
	A polinom fokszáma szeretnénk, ha nem lenne túl kicsi, és a felső határt $p$-hez viszonyítva kell megadni. Jelen megvalósításban ha $k$ a polinom fokszáma, akkor legyen $2 \leq k \leq 5\cdot\sqrt[\leftroot{-2}\uproot{2}10]{p}$.
	\subsubsection{ChaCha20 konstrukció}
	\subsubsection*{Bevezető}
	A ChaCha20 stream cipher eljárás egy belső állapoton végzett bitenkénti XOR (kizáró vagy), összeadás ($ \textrm{mod} \ 2^{32}$) és konstans távolságú forgatásokkal működik. Az állapot 16 darab 32 bites szóból tevődik össze, melynek kezdeti felépítése:
	
	\begin{table}[h]
		\centering
		\begin{tabular}{|c|c|c|c|}
			\hline
			\rowcolor{yellow}Const&Const&Const&Const\\\hline
			\rowcolor[rgb]{0.04313,0.7137,1} Key&Key&Key&Key\\\hline
			\rowcolor[rgb]{0.04313,0.7137,1} Key&Key&Key&Key\\\hline
			\cellcolor[rgb]{1,0.3686,0.3686}Counter& \cellcolor[rgb]{0,1, 0.26}Nonce&\cellcolor[rgb]{0,1, 0.26}Nonce&\cellcolor[rgb]{0,1, 0.26}Nonce\\\hline
		\end{tabular}
	\end{table}

	A konstans mezők együttes értéke az "expand 32-byte k" szöveget adja ki. A fő művelet az úgynevezett negyedkör (quarter round), ez egy 4 paraméteres eljárás, melynek minden paramétere egy 32-bites szó változója. Jelöljük ezt QR($a$, $b$, $c$, $d$)-vel, ekkor a QR($a$, $b$, $c$, $d$) eljárás (C stílusú pszeudokódként):
	\begin{center}
	$a$ += $b$; $d$ $^\wedge$= $a$; $d <<<$= $16$; \\
	$c$ += $d$; $b$ $^\wedge$= $c$; $b <<<$= $12$; \\
	$a$ += $b$; $d$ $^\wedge$= $a$; $d <<<$= $8$; \\
	$c$ += $d$; $b$ $^\wedge$= $c$; $b <<<$= $7$;
	\end{center}
	Ahol az $a$ += $b$ az $a$ és $b$ bitenkénti összeadását jelenti $\textrm{mod}\ 2^{32}$. A moduláris összeadás a megvalósításban simán összeadást jelent kettő előjel nélküli 32 bites egész szám között ahol a maradékot az overflow kezeli. Az eredmény az $a$ változóba kerül.
	
	Az $a$ $^\wedge$= $b$ a bitenkéni XOR, vagyis kizáró vagy műveletet jelöli. Az eredmény az $a$ változóba kerül.
	
	Az $a <<<$= $b$ az $a$ 32 bites változó balra forgatását jelöli $b$-vel. A $b$-vel való balra forgatás egy bitsorozat balra shiftelését (a magasabb helyi értékek felé shiftelünk) jelenti $b$-szer úgy, hogy a magas helyi értéken túlmenő bitek visszakerülnek a legkisebb helyi értékre (más néven: ciklikus shiftelés). A forgatás eredménye szintén az $a$ változóba kerül.
	\subsubsection*{Algoritmus}
	
	Az algoritmus 20 kört hajt végre, egy körben 4 negyedkört csinálva, váltogatva a körök csoportját, hogy éppen melyik indexen hajtódnak végre a negyedkörök. Kétfajta csoport van, az egyik az oszlopos, másik a diagonális kör. Ha a 16 szavas állapot 4x4-es mátrixát az alábbiak szerint indexeljük:
	\begin{table}
		[h]
		\centering
		\begin{tabular}{|c|c|c|c|}
			\hline
			0&1&2&3\\\hline
			4&5&6&7\\\hline
			8&9&10&11\\\hline
			12&13&14&15 \\\hline
		\end{tabular}
	\end{table}
	\\
	Akkor a két fajta negyedkör az alábbiak szerint hajtódik végre:
	\begin{enumerate}
		\bfseries\item Oszlopos kör \\
		\normalfont Mindegyik negyedkör az egyik oszlopban lévő elemeken hajtódik végre, sorban az 1. oszloptól kezdve, tehát a pszeudokódja (a fent definiált QR eljárással):
		\\QR(0, 4, 8, 12); \\
		QR(1, 5, 9, 13); \\
		QR(2, 6, 10, 14); \\
		QR(3, 7, 11, 15);
		\bfseries \item Diagonális kör \\
		\normalfont Mindegyik negyedkör az egyik diagonálison álló elemeken hajtódik végre a főátlótól jobbra haladva. A pszeudokódja:
	    \\QR(0, 5, 10, 15); \\
	    QR(1, 6, 11, 12); \\
	    QR(2, 7,  8, 13); \\
	    QR(3, 4,  9, 14); \\
	\end{enumerate}
	Az oszlopos körök csoportját megfeleltetjük a páratlan, a diagonális körökét a páros sorszámú körökkel. A 20 kör tehát 10 darab dupla körből áll, ahol az első kör oszlopos, a második diagonális. Tehát 10 iterációnyi dupla kört kell futtatni, ezzel együtt a ChaCha20 precíz algoritmusa:
	\begin{enumerate}
		\item A ChaCha20 állapotát elő kell állítani, ezt a programban számgenerálással, illetve manuálisan lehet megadni. A kezdeti belső ChaCha állapot a fent leírt mátrixos alakban legyen $I$, ezt le kell másolnunk egy átmeneti $X$ állapotba. $I[i]$ az $I$ állapot $i$. indexű elemére hivatkozik a fenti indexeléssel.
		\item Legyen $i := 1$, majd hajtsunk végre egy dupla kört az $X$, másolt állapoton, vagyis:
		\begin{enumerate}
			\item Oszlopos kör: \\
			QR($X[0]$, $X[4]$, $X[8]$, $X[12]$); \\
			QR($X[1]$, $X[5]$, $X[9]$, $X[13]$); \\
			QR($X[2]$, $X[6]$, $X[10]$, $X[14]$); \\
			QR($X[3]$, $X[7]$, $X[11]$, $X[15]$);
			\item Diagonális kör:
			\\QR($X[0]$, $X[5]$, $X[10]$, $X[15]$); \\
			QR($X[1]$, $X[6]$, $X[11]$, $X[12]$); \\
			QR($X[2]$, $X[7]$, $X[8]$, $X[13]$); \\
			QR($X[3]$, $X[4]$, $X[9]$, $X[14]$); \\
			\item Inkrementáljuk $i$-t, ha $i <= 10$, akkor ugorjunk vissza (a)-ra.
		\end{enumerate}
		\item Egy menet kimenete egy 512 bites ($16\cdot32$, az állapot mérete) sorozat, legyen ez $O$, ugyanolyan felépítéssel mint az $X$ és $I$ (16 méretű 32 bites szavak tömbje). Legyen ekkor $\forall \ i \in \{0, 1,..., 15\}$-re:
		$$O[i] = X[i] + I[i]$$
		Ezzel előállítottuk a kimeneti $O$ bitsorozatot.
		\item Növeljük az belső állapot számlálóját, vagyis inkrementáljuk $I[12]$-őt.
	\end{enumerate}
	Nyilvánvalóan a tetszőlegesen hosszú stream generáláshoz az $O$ bitsorozat nem elég. A fenti algoritmust $ChaChaRound()$-al jelölve a stream generálás algoritmusa a következőféleképpen alakul:
	\begin{enumerate}
		\item Legyen $l$ a generálandó sorozat mérete, ezt fogjuk használni az algoritmusban arra is, hogy számon tartsuk még mennyi bájtnyi adatot kell előállítani. A kimenetnek megfelelő bitsorozat legyen $O$.
		\item Amíg $l \geq 64$, addig:
		\begin{enumerate}
			\item Hajtsuk végre a $ChaChaRound()$ eljárást. Az eljárás teljes kimenetét fűzzük az $O$ kimeneti bitsorozathoz.
			\item Csökkentsük $l$-t 64-el, vagyis legyen $l := l - 64$.
		\end{enumerate}
		\item Ha $l > 0$: Még egyszer hajtsuk végre a $ChaChaRound()$ eljárást. A maradék $l$ bájtnyi kimenetet az eljárás eredményeként kapott első $l$ bájtjából állítjuk össze, a mátrixos indexeléssel élve. Ha már egy teljes szónyi adat nem fér bele $O$-ba, akkor az utolsó szóból ami benne van az első $l$ bájtban már csak az első $l \ \textrm{mod} \ 4$ bájtot fűzzük a kimenethez.
	\end{enumerate} 
	Jelen körülmények között extra követelményt nem támasztunk az algoritmus felé, de a kezdeti állapotot kifejezetten kell ellenőrizni, ha sokat akarunk az algoritmussal generálni. Ha sok kört fut be ugyanaz az állapot, akkor a számláló körbeér, és ismétlődni fognak a generált értékek. Ekkor mindenképpen új állapotot kell választani, tehát maximálisan $2^{32}-1$ darab $ChaChaRound()$ hívás legyen mielőtt új állapotot készítünk.
	
	Ezen kívül a nonce részét az állapotnak minden titkosított adatvitel esetén csak egyszer használjuk egy adott kulccsal, hogy kikerüljük a visszajátszásos támadásokat.
	
	A ChaCha algoritmust még nem vizsgálták kimerítően kriptográfiailag, csak a Salsa változatot. A jelenlegi legjobb ismert támadás $2^{109}$ művelettel töri fel a 7 menetes, és $2^{250}$ művelettel a 8 menetes Salsa titkosítást.
	
	Azt is megmutatta 2013-ban Nicky Mouha és Bart Prencel [https://eprint.iacr.org/2013/328.pdf], hogy a Salsa 128-bites biztonságot jelent a differenciális kriptoanalízises támadásokkal szemben. Ezek olyan támadásokat jelentenek, mely a bemenet megváltoztatásából a kimenet változásaira próbálnak következtetéseket levonni.
	\subsection{Mértékek}
	\subsubsection{Ismertető}
	A régi próbálkozások a pszeudovéletlenség definiálására nem voltak elég jól alkalmazhatóak a gyakorlatban, mint pl.: Golomb véletlenségi posztulátumai vagy Knuth pszeudovéletlenségi definíciója. Ezek a definíciók ténylegesen jó pszeudovéletlen sorozatokat eredményeznek, de a tesztelés rendkívül költséges, és a kivitelezés is nehéz.
	
	1996-ban Christian Mauduit és Sárközy András kidolgoztak egy újféle megközelítést a pszeudovéletlenség vizsgálatára számmal mérhető mértékekkel. Ez a megközelítés eredményesnek bizonyult, belátható a mértékek eredményei alapján, hogy egy sorozat jó pszeudovéletlen tulajdonságú-e. A konstrukciókról így az is bizonyítható, hogy ténylegesen jó a generált sorozat véletlensége.
	\subsubsection{Alapfogalmak}
	Először is fontos megjegyezni, hogy az alábbiakban tárgyalt konstrukciók a bináris sorozatoknak a klasszikus $0$-ások és $1$-esek megfeleltetését annyiban változtatják meg, hogy a $0$-ákat $-1$-eseknek veszi. \\
	Legyen tehát $E=(e_1, e_2, ..., e_N) \in \{-1, 1\}^N$ egy $N$ hosszúságú bináris sorozat. A mértékek számolásánál fontos ez a megfeleltetés az egyszerű számoláshoz.
	\subsubsection*{Jól-eloszlás}
	A fenti definícióval együtt, legyen $E_N=(e_1, e_2, ..., e_N) \in \{-1, 1\}^N$ egy $N$ hosszúságú bináris sorozat. Jelölje $U(E_N, M, a, b)$ a következő értéket ($a+M\cdot b \leq N$):
	$$U(E_N, M, a, b) = \sum_{i=1}^{M}e_{a+i\cdot b}$$
	Az $U$ függvény megadja, hogy a sorozatnak az $a$ kezdőponttól $b$ lépésközzel a számok összege mennyi a sorozat végéig.
	
	Ekkor az \textbf{jól-eloszlási mértéket} definiáljuk a következőféleképpen:
	$$W(E_N) = \max_{a, b, t}{|U(E_N, t, a, b)|}$$
	Ahol $a \in \mathbb{Z}$ és $t, b \in \mathbb{N}$ úgy, hogy $1 \leq a+b \leq a+t\cdot b \leq N$.
	
	A mérték gyakorlatilag azt vizsgálja, hogy az adott lépésközökkel és kezdőpozíciókkal mi a legnagyobb kitérés az egyenlő, fele-fele $-1$, $1$ elosztástól.
	
	\subsubsection*{Normalitás}
	Legyen $E_N=(e_1, e_2, ..., e_N) \in \{-1, 1\}^N$ egy $N$ hosszúságú bináris sorozat. Az $X=(x_1, x_2, ..., x_k) \in \{-1,1\}^k$ is bináris sorozat, ahol $k, M \in \mathbb{N}$ és $1 \leq M \leq N-k+1$. Ekkor legyen $T(E_N, M, X)$ a következő:
	$$T(E_N, M, X) = |\{n : 0 \leq n < M, (e_{n+1}, e_{n+2}, ..., e_{n+k}) = X\}|$$
	Vagyis azon pozíciók száma az $M$. kezdőpozícióig, amiktől vett $k$ hosszú részsorozat megegyezik az $X$ részsorozattal. Gyakorlatilag az előfordulásait számolja meg $X$-nek, $M$-ig.
	
	Ekkor a \textbf{k-ad rendű normalitás mértéket} a következőképpen definiáljuk:
	$$N_k(E_N) = \max_{x \in \{-1, 1\}^{k}}\max_{0 < M \leq N-k+1}{\left| T(E_N, M, x) - \frac{M}{2^k}\right|}$$
	Vagyis a $k$ hosszú részsorozatok esetében az átlagos előfordulási számtól való legnagyobb eltérést fejezi ez ki. \par
	Így már definiálható a \textbf{normalitás mérték}, mely a következőféleképpen alakul:
	$$N(E_N) = \max_{k \leq \lfloor\log_2{N}\rfloor}{N_k(E_N)}$$
	Vagyis a legfeljebb $\lfloor\log_2{N}\rfloor$ rendű $k$-ad rendű normalitás mértékek maximumát vesszük.
	\subsubsection*{Korreláció}
	Legyen $E_N=(e_1, e_2, ..., e_N) \in \{-1, 1\}^N$ egy $N$ hosszúságú bináris sorozat. $D = (d_1, ..., d_k) \in \mathbb{N}^k$ és $M \in \mathbb{N}$ úgy, hogy $0 \leq d_1 < d_2 < ... < d_k < M+d_k \leq N$ esetén legyen $V(E_N, M, D)$ a következő mennyiség:
	$$V(E_N, M, D) = \sum_{n=0}^{M-1}{\prod_{i=1}^{k}{e_{n+d_i}}}$$
	Vagyis a $V$ függvény összegzi az összes lehetséges pozícióból kezdve a $d_1, ..., d_k$-val eltolt pozícióban lévő elemek szorzatát.
	\par
	Ekkor definiálhatjuk a \textbf{k-ad rendű korreláció mértéket} a következőféleképpen:
	$$C_k(E_N) = \max_{M, D}{|V(E_N, M, D)|}$$
	ahol $D \in \mathbb{N}^k$ és $M \in \mathbb{N}$ úgy ahogy előbb a $V$ függvény definiálásakor voltak állítva kritériumok, vagyis $D = (d_1, ..., d_k)$ esetén $0 \leq d_1 < d_2 < ... < d_k < M+d_k \leq N$. Ez a kifejezés azt keresi gyakorlatilag, hogy melyik az az ugrássorozat, amivel a legnagyobbnak tűnik az összefüggés mértéke. 
	\par
	Így a \textbf{korreláció mértéket} a következőféleképpen definiáljuk:
	$$C(E_N) = \max_{k \leq \lfloor\log_2{N}\rfloor}{C_k(E_N)}$$
	Vagyis a legfeljebb $\lfloor\log_2{N}\rfloor$-ed rendű korrelációk maximuma.
	\subsubsection*{Követelmények}
	Ezzel egy bináris sorozat esetén a pszeudovéletlenségéhez támasztott tulajdonságokat követeljük meg, melyek a következők (a régi, klasszikus követelmények szerint):
	\begin{enumerate}
		\bfseries \item Normalitás:
		\normalfont	Legyen $E: \mathbb{N} \to \{e_1, ..., e_N\}$, és $E(n) = e_{1+(n \mod N)}$, így $E$ egy végtelenített bináris sorozat, és tetszőleges $E_N$ esetén kiterjeszthető a végtelenítés. Ekkor $E$-t felfoghatjuk sorozatként is: $E=(e_1, ..., e_N, e_1, e_2, ..., e_N, e_1, ...)$, így ha a sorozatra:
		$$\left|T(E, M, X) - \frac{M}{2^k}\right| = o(M)$$
		minden rögzített $k$ és $X$ esetén, ahol $M \to \infty$, akkor $E$ \textbf{normális}.
		\par
		\bfseries \item Jól-eloszlás:
		\normalfont A fenti $E$ végtelenített sorozattal, $E$ \textbf{jól-eloszlású}, ha:
		$$U(E, M, a, b) = o (M)$$
		minden fix $a$ és $b$ esetén, ahol $M \to \infty$.
		\bfseries \item Kis többszörös korreláció:
		\normalfont $E$ végtelenített sorozat, $E$-nek \textbf{kicsi a többszörös korrelációja}, ha:
		$$V(E, M, D) = o(M)$$
		minden rögzített $D$ esetén, ahol $M \to \infty$. \par
		Új követelések pedig:
		\item Olyan valós értékű függvényekkel szeretnénk jellemezni a a sorozatok pszeudovéletlenségét, melyek az összes véges hosszúságú bináris sorozaton értelmezve vannak. Így bármely két sorozat összehasonlítható az eredmények szerint.
		\item A mértékeket jó tulajdonságú sorozatok esetén tudjuk becsülni.
		\item Legyen értelmezve a mértékek alacsonyabb szintje is, vagyis a pszeudovéletlen jelleg kimutatható legyen. 
	\end{enumerate}
	A legfontosabb mértékek között vannak a fent tárgyalt jól-eloszlás-, k-ad rendű normalitás-, normalitás-, k-ad rendű korreláció- és korreláció-mértékek. Ezen kívül fontos még a \textbf{k-ad rendű kombinált mérték}, ez a jól-eloszlást és k-ad rendű korrelációt ötvözi és az ehhez tartozó \textbf{kombinált mérték}, mely a normalitáséhoz hasonlóan adódik a $k$-ad rendű mértékből.
	
	Adott $E_N$ sorozat esetén akkor beszélünk jó pszeudovéletlen tulajdonságú sorozatról, ha mind a jól-eloszlás és a k-ad rendű korreláció mérték kicsi $N$ függvényében. Mauduit, Sárközy és Cassiagne bebizonyították, hogy majdnem minden $E_N \in \{-1, 1\}^N$ sorozatra a fent említett érték $O(\sqrt{N}(\log{N})^c)$ nagyságrendű, ahol $c$ konstans.
	\subsection{One-time pad titkosítás}
	\subsubsection{Ismertető}
	A One-time pad titkosítás működése kifejezetten egyszerű. Kapunk egy kódolandó bitsorozatot (üzenet), majd ezt egy titkosító kulcs bitsorozattal (mely ugyanolyan hosszú mint az üzenet hossza) bitenkénti kizáró vagy művelettel (XOR) titkosítjuk. Az eredményként kapott bitsorozat a titkosított üzenet. Ezt az üzenetet elküldhetjük a másik félnek, és ha ezen a másik fél elvégzi ugyanezt az eljárást ugyanazzal a kulccsal, akkor visszakapja az eredeti üzenetet.
	
	Picivel formálisabban: Legyen $E_N = (e_1, e_2, ..., e_N)$ a kódolandó üzenetet reprezentáló bitsorozat, illetve $X_N = (x_1, x_2, ..., x_N)$ a titkosításhoz használt kulcs bitsorozata. Ekkor, a one-time pad titkosítás eredménye az a $B_N = (b_1, b_2, ..., b_N)$ bitsorozat, melyre:
	$$B_N = E_N \oplus X_N$$
	Ahol az $\oplus$ a bitenkénti kizáró vagy művelet, vagyis elemenként a $B_N$ elemeire: $$b_i = e_i \oplus x_i \quad \forall i \in \{1, 2, ... N\}$$
	\textbf{Példa}: Legyen $N = 7$, és $E_7 = (1, 0, 1, 0, 1, 1, 0)$ és $X_7 = (0, 1, 1, 0, 0, 1, 0)$. Ekkor: $B_7 = E_7 \oplus X_7$, vagyis: \\
	\begin{center}
		\begin{tabular}{c@{\,}c@{\,}c@{\,}c@{\,}c@{\,}c@{\,}c@{\,}c}
			& 1 & 0 & 1 & 0 & 1 & 1 & 0 \\
			$\oplus$ & 0 & 1 & 1 & 0 & 0 & 1 & 0  \\
			\hline
			& 1 & 1 & 0 & 0 & 1 & 0 & 0 \\
		\end{tabular}
	\end{center}
	A továbbítandó, titkosított üzenet tehát a $B_7 = (1, 1, 0, 0, 1, 0, 0)$ bitsorozat.
	 
	A felhasználói dokumentációban már említve volt, hogy valamilyen karaktertábla segítségével kódolja a program a betűket, hogy a titkosításnak hétköznapi felhasználásban is lássuk az alkalmazást. A cél tehát, hogy a kapott üzenetet (szöveget) át kell alakítani bitsorozattá amivel aztán elvégezhető a titkosítás, majd a titkosított üzenetet is vissza kell tudni alakítani szöveges formába.
	
	A naiv elképzelés az volt, hogy valamilyen már létező fix hosszúságú karakterkódolást használjon a program, de ezt nem lehetett megvalósítani mivel mindenféle kódolásban vannak vezérlőkarakterek. Ezeket a képernyőn nem tudjuk szabályosan megjeleníteni, mivel ezeknek nem célja az olvasható szöveg reprezentációja. A bemenetben még ez nem is okoz gondot, de amikor lefut a titkosítás, akkor nincs garancia arra, hogy olyan karakter lesz az eredmény amely megjeleníthető. Sőt, minden karakterre van olyan bitsorozat, amellyel vett kizáró vagy eredménye vezérlőkarakter.
	
	Kell tehát egy olyan karaktertábla, mely tartalmazza a magyar szövegekben széles körben használt karaktereket, így született meg a jelenlegi karakterkészlet, mely a magyar billentyűzettel leírható karakterek nagy részét tartalmazza. Összesen 128 karaktert tartalmaz a program, így fixen 7 biten tudunk kódolni minden használt karaktert. A kódolás bővíthető, de ekkor mindig valamilyen hatványát kell használni a 2-nek, hogy fix hosszúságú biten tudjuk kódolni, pl. a 128-ról az első lehetséges bővítés a 256 karakteres tábla lenne.
	
	Kód szintjén a karaktereket a Unicode kódolásból vett számukkal azonosítjuk, és ehhez rendeljük a 7 hosszúságú bitsorozatot. A program fájlból építi fel ezt a szótárat, szóval egyedi kódolás is megadható, a dictionary.txt fájl megfelelő módosításával (a program egyenlőre csak a 7 bites kódolást kezeli, tehát a karaktertábla méretét bővíteni csak a fájl módosításával nem lehet).
	\begin{table}
	\begin{center}
		\begin{tabular}{|>{\columncolor[gray]{0.8}}c|c|>{\columncolor[gray]{0.8}}c|c|>{\columncolor[gray]{0.8}}c|c|>{\columncolor[gray]{0.8}}c|c|}
			\hline 0 & 0000000 & 1 & 0000001 & 2 & 0000010 & 3 & 0000011 \\
			\hline 4 & 0000100 & 5 & 0000101 &
			6 & 0000110 & 7 & 0000111 \\
			\hline 8 & 0001000 & 9 & 0001001 &
			 ö & 0001010 & ü & 0001011 \\
			\hline ó & 0001100 & q & 0001101 &
			w & 0001110 & e & 0001111 \\
			\hline r & 0010000 & t & 0010001 &
			z & 0010010 & u & 0010011 \\
			\hline i & 0010100 & o & 0010101 & 
			p & 0010110 & ő & 0010111 \\ 
			\hline ú & 0011000 & a & 0011001 &
			s & 0011010 & d & 0011011 \\
			\hline f & 0011100 & g & 0011101 &
			h & 0011110 & j & 0011111 \\
			\hline k & 0100000 & l & 0100001 &
			é & 0100010 & á & 0100011 \\
			\hline ú & 0100100 & í & 0100101 &
			y & 0100110 & x & 0100111 \\
			\hline c & 0101000 & v & 0101001 &
 			b & 0101010 & n & 0101011 \\
			\hline m & 0101100 & , & 0101101 &
			. & 0101110 & - & 0101111 \\
			\hline § & 0110000 & ' & 0110001 &
			" & 0110010 & + & 0110011 \\
			\hline ! & 0110100 & \% & 0110101 &
			/ & 0110110 & = & 0110111 \\
			\hline ( & 0111000 & ) & 0111001 &
			Ö & 0111010 & Ü & 0111011 \\
			\hline Ó & 0111100 & Q & 0111101 &
			W & 0111110 & E & 0111111 \\
			\hline R & 1000000 & T & 1000001 &
			Z & 1000010 & U & 1000011 \\
			\hline I & 1000100 & O & 1000101 &
			P & 1000110 & Ő & 1000111 \\    
		    \hline Ú & 1001000 & A & 1001001 &
		    S & 1001010 & D & 1001011 \\
		    \hline F & 1001100 & G & 1001101 &
		    H & 1001110 & J & 1001111 \\
		   \hline K & 1010000 & L & 1010001 &
		   É & 1010010 & Á & 1010011 \\
		  \hline Ű & 1010100 & Í & 1010101 &
		  Y & 1010110 & X & 1010111 \\
		  \hline C & 1011000 & V & 1011001 &
		  B & 1011010 & N & 1011011 \\
		  \hline M & 1011100 & ? & 1011101 
		  & : & 1011110 & \_ & 1011111 \\
		  \hline $\sim$ & 1100000 & ˇ & 1100001 &
		  \textasciicircum & 1100010 & ˘ & 1100011 \\
		  \hline ` & 1100100 & ´ & 1100101 &
		  ˝ & 1100110 & ¨ & 1100111 \\
		  \hline ¸ & 1101000 & $\setminus$ & 1101001 &
		  $|$ & 1101010 & Ä & 1101011 \\
		  \hline \euro & 1101100 & ÷ & 1101101 &
		  × & 1101110 & ä & 1101111 \\
		  \hline đ & 1110000 & [ & 1110001 &
		  ] & 1110010 & ł & 1110011 \\
		  \hline Ł & 1110100 & \$ & 1110101&
		   $<$ & 1110110 & $>$ & 1110111 \\
		  \hline \# & 1111000 & \& & 1111001 &
		   @ & 1111010 & \{ & 1111011 \\
			\hline \} & 1111100 & ; & 1111101 &
		 * & 1111110 & & 1111111 \\ \hline
		\end{tabular}
	\end{center}
	 \begin{center}
	 	\textit{Ábra: A teljes kódolás}
	 \end{center}
	\end{table}
\newpage
\subsubsection{Algoritmus}
A fentiek figyelembevételével az általános algoritmus a következőképpen alakul:
\begin{enumerate}
	\item Hozzuk létre a kódolást reprezentáló adatszerkezetet. Ajánlott valamilyen formájú hash táblát használni, hogy az elemek elérése konstans műveletigényű legyen. Jelen megvalósításban ez két std::unordered\_map-al valósul meg, hogy mindkét irányból el tudjuk érni az adatokat.
	\item Adjuk meg a titkosítandó üzenetet és a titkosításhoz kulcsként használt bitsorozatot.
	\item Az így kapott két bitsorozattal végezzünk el egy bitenkénti kizáró vagy műveletet. Ha a kulcs hosszabb mint a titkosítandó sorozat, akkor a kulcsból vehetjük az első $l$ bitet, ahol $l$ a titkosítandó bitsorozat hossza. Az eredmény a titkosított üzenet bitsorozatos alakja.
	\item Az előbb kiszámított sorozatot alakítsuk vissza a kódolás szerint karakteres formába. Az eredmény a titkosított üzenet szöveges alakja.
\end{enumerate}
\subsubsection{Extra követelmények}
A titkosítandó szöveg nem tartalmazhat a kódolásban nem szereplő karaktert.  Ezen felül a titkosításhoz használt kulcssorozat hossza legalább akkora kell legyen mint a titkosítandó üzenet bitsorozatos alakban vett hossza.
\end{document}