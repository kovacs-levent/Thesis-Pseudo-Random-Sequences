\documentclass[12pt]{article}
\usepackage[document]{ragged2e}
\usepackage{geometry, array, graphicx, wrapfig, float, amsmath}
\usepackage[thinlines]{easytable}
\graphicspath{ {./pics/} }
\geometry{a4paper, inner=3.5cm, outer=2.5cm, bottom=2.5cm, top=2.5cm}
\renewcommand{\baselinestretch}{1.5}
\begin{document}
	\begin{titlepage}
		\begin{minipage}{0.36\textwidth}
			\begin{figure}[H]
				\includegraphics[scale=0.1]{elte_cimer_szines.jpg}
			\end{figure}
		\end{minipage}
		\begin{minipage}{0.60\textwidth}
			\begin{center}
				{\Large Eötvös Loránd Tudományegyetem \\
					Informatikai Kar \\
					Komputeralgebra Tanszék \\}
			\end{center}
		\end{minipage}
		\\[0.3\baselineskip]
		\noindent\makebox[\linewidth]{\rule{\textwidth}{0.5pt}}
		\vspace*{\fill}
		\centering
		\vspace*{0.5cm}
		
		\huge\bfseries
		Pszeudovéletlen sorozatok mértékei és konstrukciói		
		\vspace*{0.5cm}
		
		\vspace*{\fill}
		\begin{TAB}(r,0.5cm,0.5cm)[9pt]{cc}{cccc}
			\normalfont \normalsize \textsl{ Szerző: } & \normalfont \normalsize \textsl{ Témavezető: } \\
			\normalsize Kovács Levente & \normalsize Dr. Tóth Viktória \\
			\normalfont \normalsize programtervező informatikus BSc & \normalfont \normalsize egyetemi adjunktus \\
			\normalsize & \normalfont \normalsize Komputeralgebra Tanszék
		\end{TAB}
		\\
		\large \normalfont Budapest, 2019
	\end{titlepage}
	
\end{document}